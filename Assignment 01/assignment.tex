\documentclass{article}
\usepackage[a4paper, left=2cm, right=2cm, top=3cm, bottom=3cm, headheight=35pt, includehead]{geometry}

\usepackage[english]{babel}
\usepackage[utf8]{inputenc}
\usepackage{fancyhdr}

\usepackage{enumitem}
\usepackage{amsmath}
\usepackage{mathtools}
\usepackage{listings}


\lstset{mathescape=true,
		morekeywords={if,else,and,then,for,return}}

\pagestyle{fancy}
\fancyhf{}
\lhead{\today \\ Andrés Montoya, 405409 \\ Laura Koch, 406310 }
\chead{Introduction to Artificial Intelligence \\ Assignment 01}
\rhead{ Lennart Holzenkamp, 407761 \\ Simon Michau, 406133 \\ Til Mohr, 405959}

\begin{document}

\section*{Exercise 1.1}
\begin{enumerate}[label=(\alph*)]
	\item	Yes, the elevator control can be regarded as an agent, as it follows the architecture of rational agents.

			It has sensors (buttons) to percieve (button status) the environment (elevator + floors) and will operate (move up/down + open/close doors) based on that input, with the goal of transporting people.

			The elevator control agent is reflexive with an internal state, as it always remembers the current floor the elevator is in.

			Depending on the implementation of $\phi_v^u$ the agent may be goal-based or utility-based. If $\phi_v^u$ simply selects a random floor to be served next, it is just goal-based. If $\phi_v^u$ selects the next floor to be served while minimizing waiting time for example, it is utility-based.
	\item	\begin{itemize}
				\item	Persons transported per hour
				\item	Average waiting time for buttons pressed in the elevator
				\item	Average waiting time for buttons pressed on a floor
			\end{itemize}
	\item	The following function $\phi_v^u$ operates best when M/W are implemented as a stack.
\begin{lstlisting}
next $\leftarrow$ f
m $\leftarrow$ M[0]
w $\leftarrow$ W[0]

if m is null and w is null then
	return f
if m is null and w is not null then
	next $\leftarrow$ w
if m is not null and w is null then
	next $\leftarrow$ m
if m is not null and w is not null then
	if |m-f|>|w-f| then
		next $\leftarrow$ w
	else
		next $\leftarrow$ m

for x in M $\cup$ W
	if |x-f|<|next-f| then
		next $\leftarrow$ x

return x
\end{lstlisting}
\end{enumerate}

\section*{Exercise 1.2}
\begin{enumerate}[label=(\alph*)]
	\item	\begin{itemize}
				\item	semi-accessible: Information on number of people waiting on each floor and number of people inside the elevator could optimize the agent
				\item	deterministic:
				\item	episodc: choice of action only depends on current state
				\item	dynamic: new buttons can be pressed while deciding
				\item	discrete:
			\end{itemize}
	\item	\begin{itemize}
				\item
			\end{itemize}
	\item	\begin{itemize}
				\item
			\end{itemize}
\end{enumerate}

\section*{Exercise 1.3}
\begin{enumerate}[label=(\alph*)]
	\item
	\item
	\item
\end{enumerate}

\section*{Exercise 1.4}

\end{document}